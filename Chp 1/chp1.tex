%%%%%%%%%%%%%%%%%%%%%%%%%%%%%%%%%%%%
% Slide options
%%%%%%%%%%%%%%%%%%%%%%%%%%%%%%%%%%%%

% Option 1: Slides with solutions

\documentclass[slidestop,compress,mathserif]{beamer}\usepackage[]{graphicx}\usepackage[]{color}
%% maxwidth is the original width if it is less than linewidth
%% otherwise use linewidth (to make sure the graphics do not exceed the margin)
\makeatletter
\def\maxwidth{ %
  \ifdim\Gin@nat@width>\linewidth
    \linewidth
  \else
    \Gin@nat@width
  \fi
}
\makeatother

\definecolor{fgcolor}{rgb}{0.345, 0.345, 0.345}
\newcommand{\hlnum}[1]{\textcolor[rgb]{0.686,0.059,0.569}{#1}}%
\newcommand{\hlstr}[1]{\textcolor[rgb]{0.192,0.494,0.8}{#1}}%
\newcommand{\hlcom}[1]{\textcolor[rgb]{0.678,0.584,0.686}{\textit{#1}}}%
\newcommand{\hlopt}[1]{\textcolor[rgb]{0,0,0}{#1}}%
\newcommand{\hlstd}[1]{\textcolor[rgb]{0.345,0.345,0.345}{#1}}%
\newcommand{\hlkwa}[1]{\textcolor[rgb]{0.161,0.373,0.58}{\textbf{#1}}}%
\newcommand{\hlkwb}[1]{\textcolor[rgb]{0.69,0.353,0.396}{#1}}%
\newcommand{\hlkwc}[1]{\textcolor[rgb]{0.333,0.667,0.333}{#1}}%
\newcommand{\hlkwd}[1]{\textcolor[rgb]{0.737,0.353,0.396}{\textbf{#1}}}%
\let\hlipl\hlkwb

\usepackage{framed}
\makeatletter
\newenvironment{kframe}{%
 \def\at@end@of@kframe{}%
 \ifinner\ifhmode%
  \def\at@end@of@kframe{\end{minipage}}%
  \begin{minipage}{\columnwidth}%
 \fi\fi%
 \def\FrameCommand##1{\hskip\@totalleftmargin \hskip-\fboxsep
 \colorbox{shadecolor}{##1}\hskip-\fboxsep
     % There is no \\@totalrightmargin, so:
     \hskip-\linewidth \hskip-\@totalleftmargin \hskip\columnwidth}%
 \MakeFramed {\advance\hsize-\width
   \@totalleftmargin\z@ \linewidth\hsize
   \@setminipage}}%
 {\par\unskip\endMakeFramed%
 \at@end@of@kframe}
\makeatother

\definecolor{shadecolor}{rgb}{.97, .97, .97}
\definecolor{messagecolor}{rgb}{0, 0, 0}
\definecolor{warningcolor}{rgb}{1, 0, 1}
\definecolor{errorcolor}{rgb}{1, 0, 0}
\newenvironment{knitrout}{}{} % an empty environment to be redefined in TeX

\usepackage{alltt}
\newcommand{\soln}[1]{\textit{#1}}
\newcommand{\solnGr}[1]{#1}

% Option 2: Handouts without solutions

%\documentclass[11pt,containsverbatim,handout]{beamer}
%\usepackage{pgfpages}
%\pgfpagesuselayout{4 on 1}[letterpaper,landscape,border shrink=5mm]
%\newcommand{\soln}[1]{ }
%\newcommand{\solnGr}{ }

%%%%%%%%%%%%%%%%%%%%%%%%%%%%%%%%%%%%
% Style
%%%%%%%%%%%%%%%%%%%%%%%%%%%%%%%%%%%%

\input{../lec_style.tex}


%%%%%%%%%%%%%%%%%%%%%%%%%%%%%%%%%%%%
% Preamble
%%%%%%%%%%%%%%%%%%%%%%%%%%%%%%%%%%%%

\title[Chp 1: Intro. to data]{Chapter 1: Introduction to data}
\author{OpenIntro Statistics, 3rd Edition}
\institute{$\:$ \\ {\footnotesize Slides developed by Mine \c{C}etinkaya-Rundel of OpenIntro. \\
The slides may be copied, edited, and/or shared via the \webLink{http://creativecommons.org/licenses/by-sa/3.0/us/}{CC BY-SA license.} \\
Some images may be included under fair use guidelines (educational purposes).}}
\date{}

%%%%%%%%%%%%%%%%%%%%%%%%%%%%%%%%%%%%
% Begin document
%%%%%%%%%%%%%%%%%%%%%%%%%%%%%%%%%%%%
\IfFileExists{upquote.sty}{\usepackage{upquote}}{}
\begin{document}


%%%%%%%%%%%%%%%%%%%%%%%%%%%%%%%%%%%%
% Title page
%%%%%%%%%%%%%%%%%%%%%%%%%%%%%%%%%%%%

{
\addtocounter{framenumber}{-1} 
{\removepagenumbers 
\usebackgroundtemplate{\includegraphics[width=\paperwidth]{../OpenIntro_Grid_4_3-01.jpg}}
\begin{frame}

\hfill \includegraphics[width=20mm]{../oiLogo_highres}

\titlepage

\end{frame}
}
}


%%%%%%%%%%%%%%%%%%%%%%%%%%%%%%%%%%%%
% Sections
%%%%%%%%%%%%%%%%%%%%%%%%%%%%%%%%%%%%


\section{Introduction}

\begin{frame}[fragile]

\frametitle{Goals of The Course}

After theory (what you learnin substantive courses) and after design (what you learn in 700), comes analysis (what you learn here and in 702) - telling your story with data.  

\begin{itemize}
	\item Does what you predict actually happen? 

	\item Do your data/does your design meet the assumptions of the proceure you're using?

	\item How big are the effects? 

	\item How do you convey these results to your readers? 
\end{itemize}
	
\end{frame}



\input{1-1_case_study/1-1_case_study}
\input{1-2_data_basics/1-2_data_basics}
\input{1-3_data_collection_principles/1-3_data_collection_principles}
\input{1-4_obs_studies_sampling/1-4_obs_studies_sampling}
%%%%%%%%%%%%%%%%%%%%%%%%%%%%%%%%%%%%

\section{Experiments}

%%%%%%%%%%%%%%%%%%%%%%%%%%%%%%%%%%%%

\begin{frame}
\frametitle{Principles of experimental design}

\begin{enumerate}

\item \hl{Control:} Compare treatment of interest to a control group.

\item \hl{Randomize:} Randomly assign subjects to treatments, and randomly sample from the population whenever possible.

\item \hl{Replicate:} Within a study, replicate by collecting a sufficiently large sample. Or replicate the entire study.

\item \hl{Block:} If there are variables that are known or suspected to affect the response variable, first group subjects into \hl{blocks} based on these variables, and then randomize cases within each block to treatment groups.

\end{enumerate}

\end{frame}

%%%%%%%%%%%%%%%%%%%%%%%%%%%%%%%%%%%%

% \begin{frame}
% \frametitle{More on blocking}


% \twocol{0.25}{0.75}
% {
% \begin{center}
% \includegraphics[width=\textwidth]{1-3_data_collection_principles/figures/gu}
% \end{center}
% }
% {
% \begin{itemize}
% \item We would like to design an experiment to investigate if energy gels makes you run faster:

% \pause

% \begin{itemize}
% \item Treatment: energy gel
% \item Control: no energy gel
% \end{itemize}

% \pause

% \item It is suspected that energy gels might affect pro and amateur athletes differently, therefore we block for pro status:

% \pause

% \begin{itemize}
% \item Divide the sample to pro and amateur
% \item Randomly assign pro athletes to treatment and control groups
% \item Randomly assign amateur athletes to treatment and control groups
% \item Pro/amateur status is equally represented in the resulting treatment and control groups
% \end{itemize}
% \end{itemize}
% }

% \pause

% \dq{Why is this important? Can you think of other variables to block for?}

% \end{frame}

% %%%%%%%%%%%%%%%%%%%%%%%%%%%%%%%%%%%%

% \begin{frame}
% \frametitle{Practice}

% \pq{A study is designed to test the effect of light level and noise level on exam performance of students. The researcher also believes that light and noise levels might have different effects on males and females, so wants to make sure both genders are equally represented in each group. Which of the below is correct?}

% \begin{enumerate}[(a)]
% \item There are 3 explanatory variables (light, noise, gender) and 1 response variable (exam performance)
% \solnMult{There are 2 explanatory variables (light and noise), 1 blocking variable (gender), and 1 response variable (exam performance)}
% \item There is 1 explanatory variable (gender) and 3 response variables (light, noise, exam performance)
% \item There are 2 blocking variables (light and noise), 1 explanatory variable (gender), and 1 response variable (exam performance)
% \end{enumerate}

% \end{frame}

% %%%%%%%%%%%%%%%%%%%%%%%%%%%%%%%%%%%%

% \begin{frame}
% \frametitle{Difference between blocking and explanatory variables}

% \begin{itemize}

% \item Factors are conditions we can impose on the experimental units.

% \item Blocking variables are characteristics that the experimental units come with, that we would like to control for.

% \item Blocking is like stratifying, except used in experimental settings when randomly assigning, as opposed to when sampling.

% \end{itemize}

% \end{frame}

% %%%%%%%%%%%%%%%%%%%%%%%%%%%%%%%%%%%%

% \begin{frame}
% \frametitle{More experimental design terminology...}

% \begin{itemize}

% \item \hl{Placebo:} fake treatment, often used as the control group for medical studies

% \item \hl{Placebo effect:} experimental units showing improvement simply because they believe they are receiving a special treatment

% \item \hl{Blinding:} when experimental units do not know whether they are in the control or treatment group

% \item \hl{Double-blind:} when both the experimental units and the researchers who interact with the patients do not know who is in the control and who is in the treatment group

% \end{itemize}

% \end{frame}

% %%%%%%%%%%%%%%%%%%%%%%%%%%%%%%%%%%%%

% \begin{frame}
% \frametitle{Practice}

% \pq{What is the main difference between observational studies and experiments?}

% \begin{enumerate}[(a)]
% \item Experiments take place in a lab while observational studies do not need to.
% \item In an observational study we only look at what happened in the past.
% \solnMult{Most experiments use random assignment while observational studies do not.}
% \item Observational studies are completely useless since no causal inference can be made based on their findings.
% \end{enumerate}

% \end{frame}

%%%%%%%%%%%%%%%%%%%%%%%%%%%%%%%%%%%%

\begin{frame}
\frametitle{Random assignment vs. random sampling}

\begin{center}
\includegraphics[width=\textwidth]{1-5_experiments/figures/random_sample_assignment}
\end{center}

\end{frame}

%%%%%%%%%%%%%%%%%%%%%%%%%%%%%%%%%%%%
% \input{1-6_numerical_data/1-6_numerical_data}
% \input{1-7_categorical_data/1-7_categorical_data}
% \input{1-8_gender_discrimination/1-8_gender_discrimination}

\section{Introduction to R}





\subsection{Installing R and RStudio}

\begin{frame}[fragile]

\frametitle{Software Introduction: R}

\begin{itemize}
	\item \textbf{R} can be downloaded from \url{http://cran.r-project.org}
	\begin{itemize}
		\item For Windows, click on \verb"Download R for Windows", then click on \verb"base" and finally on\\ \verb"Download R 3.3.1 for Windows". 
		\begin{itemize}
			\item I would recommend MDI mode when you have the option, but it's up to you. 
		\end{itemize}
		
		\item For Mac, click on \verb"Download R for MacOS X", then on \verb"R-3.3.1-os.pkg" (where os = (snowleopard, mavericks)).  
			\begin{itemize}
				\item You should also go on that same page to the \verb"tools" link and download and install \verb"gfortran-4.2.3.dmg" and \verb"tcltk-8.5.5-x11.dmg". 
			\end{itemize}
	\end{itemize}

	\item Then, double-clicking the R icon will launch R. 

	\item We will interact with R mostly through RStudio, though. 

\end{itemize}

\end{frame}

\begin{frame}[fragile]

\frametitle{RStudio}

Rstudio is an IDE (Integrated Development Environment) for R.  It allows you to: 

\begin{itemize}
	\item write and save code 
	\item view output 
	\item manage your workspace
	\item even write papers if you want
\end{itemize} 

You can download RStudio from \url{http://www.rstudio.com/products/rstudio/download/}.

\end{frame}

\begin{frame}[fragile]

\frametitle{Entering Data in R}

You can enter data into R in a bunch of different ways, but I'll talk about a couple.  First, you can enter it directly. 

\begin{footnotesize}
\begin{knitrout}
\definecolor{shadecolor}{rgb}{0.969, 0.969, 0.969}\color{fgcolor}\begin{kframe}
\begin{alltt}
\hlstd{x} \hlkwb{<-} \hlkwd{c}\hlstd{(}\hlnum{2}\hlstd{,}\hlnum{3}\hlstd{,}\hlnum{7}\hlstd{,}\hlnum{10}\hlstd{,}\hlnum{11}\hlstd{)}
\hlkwd{mean}\hlstd{(x)}
\end{alltt}
\begin{verbatim}
## [1] 6.6
\end{verbatim}
\begin{alltt}
\hlkwd{median}\hlstd{(x)}
\end{alltt}
\begin{verbatim}
## [1] 7
\end{verbatim}
\end{kframe}
\end{knitrout}
\end{footnotesize}
\begin{itemize}

\item You could also enter it into a spreadsheet in excel, save the file as a \verb".csv", and then read into R using the \verb"read.csv". 

\item You can also read Stata datasets into R and write Stata datasets out of R, but we will save these for later. 

\end{itemize}

\end{frame}
\begin{frame}[fragile]

\frametitle{Math to Variables: R}

In R you can directly to any mathematical operation to any object that is numerical (e.g., a variable).  Using \verb"x" above, we could square it (the \verb"^" means ``to the power of'').  You can add, subtract, multiply and divide with \verb"+", \verb"-", \verb"*" and \verb"/", respectively.  You can also do multiple operations at once. 
\begin{scriptsize}
\begin{knitrout}
\definecolor{shadecolor}{rgb}{0.969, 0.969, 0.969}\color{fgcolor}\begin{kframe}
\begin{alltt}
\hlstd{x} \hlopt{+} \hlnum{3}
\end{alltt}
\begin{verbatim}
## [1]  5  6 10 13 14
\end{verbatim}
\begin{alltt}
\hlstd{x}\hlopt{/}\hlnum{2}
\end{alltt}
\begin{verbatim}
## [1] 1.0 1.5 3.5 5.0 5.5
\end{verbatim}
\begin{alltt}
\hlstd{x}\hlopt{*}\hlnum{1.5}
\end{alltt}
\begin{verbatim}
## [1]  3.0  4.5 10.5 15.0 16.5
\end{verbatim}
\begin{alltt}
\hlstd{(x}\hlopt{+}\hlnum{5}\hlstd{)}\hlopt{^}\hlnum{3}
\end{alltt}
\begin{verbatim}
## [1]  343  512 1728 3375 4096
\end{verbatim}
\end{kframe}
\end{knitrout}
\end{scriptsize}
\end{frame}

\begin{frame}[fragile]

\frametitle{Fixing Mistakes}

Let's say that we had an entry that was wrong and we wanted to fix it, let's say the 4$^{th}$ value of \verb"x" was supposed to be 9 instead of 10.  

\begin{knitrout}
\definecolor{shadecolor}{rgb}{0.969, 0.969, 0.969}\color{fgcolor}\begin{kframe}
\begin{alltt}
\hlstd{x[}\hlnum{4}\hlstd{]} \hlkwb{<-} \hlnum{9}
\hlkwd{summary}\hlstd{(x)}
\end{alltt}
\begin{verbatim}
##    Min. 1st Qu.  Median    Mean 3rd Qu.    Max. 
##     2.0     3.0     7.0     6.4     9.0    11.0
\end{verbatim}
\end{kframe}
\end{knitrout}

\end{frame}

\begin{frame}[fragile]

\frametitle{Calculating Number of Times Something Happens: R}

Calculating the number of times a condition is met in R is easy.  For example, if we wanted to know how often $x$ is bigger than 4, we could do.  
\begin{knitrout}
\definecolor{shadecolor}{rgb}{0.969, 0.969, 0.969}\color{fgcolor}\begin{kframe}
\begin{alltt}
\hlkwd{sum}\hlstd{(x} \hlopt{>} \hlnum{4}\hlstd{)}
\end{alltt}
\begin{verbatim}
## [1] 3
\end{verbatim}
\end{kframe}
\end{knitrout}

\end{frame}

\begin{frame}[fragile]

\frametitle{Calculating Percentages: R}

Calculating a proportion is just taking the number of times something happens over the total number of times it could have happened.  For example, if we wanted to know the proportion of times $x$ is bigger than 4, we need to divide the number of times $x$ is bigger than 4 by the number of values in $x$.  $x$ is bigger than four, 3 times and not bigger than four 2 times.  
\begin{knitrout}
\definecolor{shadecolor}{rgb}{0.969, 0.969, 0.969}\color{fgcolor}\begin{kframe}
\begin{alltt}
\hlkwd{sum}\hlstd{(x} \hlopt{>} \hlnum{4}\hlstd{)}\hlopt{/}\hlkwd{length}\hlstd{(x)}
\end{alltt}
\begin{verbatim}
## [1] 0.6
\end{verbatim}
\begin{alltt}
\hlkwd{mean}\hlstd{(x} \hlopt{>} \hlnum{4}\hlstd{)}
\end{alltt}
\begin{verbatim}
## [1] 0.6
\end{verbatim}
\end{kframe}
\end{knitrout}

\end{frame}




%%%%%%%%%%%%%%%%%%%%%%%%%%%%%%%%%%%%
% End document
%%%%%%%%%%%%%%%%%%%%%%%%%%%%%%%%%%%%

\end{document}
